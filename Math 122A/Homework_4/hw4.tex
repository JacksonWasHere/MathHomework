\documentclass{article}

\usepackage{jacksonmath}
\usepackage[english]{babel}
\usepackage[utf8]{inputenc}
\usepackage[final]{pdfpages}

\begin{document}
  \section{4.7 Problems: 11, 12, 15}
  \begin{enumerate}
    \item[11]
      We will consider the equation:
      \begin{align*}
        &f(z)=\frac{xy^2(x+iy)}{x^2+x^4}\\
        &f(0)=0
      \end{align*}
      Let us take the limit when $z\to0$ along any line $z=(a+bi)t$
      \begin{align*}
        \lim_{t\to0}\frac{f((a+ib)t)}{(a+ib)t}&=\frac{(at)(bt)^2(a+ib)t}{((at)^2+(at)^4)(a+ib)t}\\
        &=\frac{ab^2t^3}{a^2t^2+a^4t^4}\\
        &=\frac{ab^2t^1}{a^2+a^4t^2}\\
        &=0
      \end{align*}
      So the limit goes to zero when $z\to0$ along a straight path. But what if we take a different path to 0. We will instead let $z(t)=t^2+it$ as $z\to0$.
      \begin{align*}
        \lim_{t\to0}\frac{f(t^2+it)}{t^2+it}&=\frac{t^2t^2(t^2+it)}{(t^4+t^8)(t^2+it)}\\
        &=\frac{t^4}{t^4+t^8}\\
        &=\frac{1}{1+t^4}\\
        &=1
      \end{align*}
    \item[12]
      \begin{enumerate}[label=(\roman*)]
        \item
          First we will find $f'$
          \[f'(z)=(z^2)'=2z\]
          Then we will plug in the path
          \[f'(\gamma(t))=2(t^3+it^4)\]
          We can easily find the derivative of the path
          \[\gamma'(t)=3t^2+4it^3\]
          Now we want to find the derivative of the compostion
          \begin{align*}
            (f\gamma)'(t)&=\left((t^3+it^4)^2\right)'\\
            &=\left(t^6-t^8+2it^7\right)'\\
            &=-8t^7+14it^6+6t^5\\
            f'(\gamma(t))\gamma'(t)&=2(t^3+it^4)(3t^2+4it^3)\\
            &=2(3t^5+7it^6-4t^7)\\
            &=-8t^7+14it^6+6t^5
          \end{align*}
          So $f'(\gamma(t))\gamma'(t)=(f\gamma)'(t)$.
        \item
          First we will find $f'$
          \[f'(z)=(1/z)'=-1/z^2\]
          Then we will plug in the path
          \[f'(\gamma(t))=-1/(\cos(t)+i\sin(t))^2=-e^{-2it}\]
          We can easily find the derivative of the path
          \[\gamma'(t)=-\sin(t)+i\cos(t)=ie^{it}\]
          Now we want to find the derivative of the compostion
          \begin{align*}
            (f\gamma)'&=\left(\frac{1}{\cos(t)+i\sin(t)}\right)'\\
            &=\left(e^{-it}\right)'\\
            &=-ie^{-it}\tag{1}\\
            f'(\gamma(t))\gamma'(t)&=-e^{-2it}ie^{it}\\
            &=-ie^{-it}\tag{2}
          \end{align*}
          (1)=(2) so $f'(\gamma(t))\gamma'(t)=(f\gamma)'(t)$.
        \item
          First we will find $f'$
          \[f'(z)=\left(\sum z^n\right)'=\sum nz^{n-1}\]
          Then we will plug in the path
          \[f'(\gamma(t))=\sum n(t+it^2)^{n-1}\]
          We can easily find the derivative of the path
          \[\gamma'(t)=1+2it\]
          Now we want to find the derivative of the compostion
          \begin{align*}
            (f\gamma)'&=\left(\sum (t+it^2)^n\right)'\\
            &=\sum n(t+it^2)^{n-1}(1+2it)\\
            f'(\gamma(t))\gamma'(t)&=(1+2it)\sum n(t+it^2)^{n-1}\\
            &=\sum n(t+it^2)^{n-1}(1+2it)
          \end{align*}
          So $f'(\gamma(t))\gamma'(t)=(f\gamma)'(t)$.
      \end{enumerate}
    \item[15]
      Let's consider the series $s(z)=\sum a_nz^n, c(z)=\sum b_n,z^n$. Let's take the derivative of $s(z)$ and $c(z)$:
      \[s'(z)=\sum a_nnz^{n-1}\]
      \[c'(z)=\sum b_nnz^{n-1}\]
      We can then work out the coefficients because we know $s'(z)=c(z)$ and $c'(z)=-s(z)$:
      \begin{align*}
        \sum a_nnz^{n-1}&=\sum b_{n-1}z^{n-1}\tag{1}\\
        \sum b_nnz^{n-1}&=\sum -a_{n-1}z^{n-1}\tag{2}\\
      \end{align*}
      If we combine (1) and (2) to solve for $a_n$
      \begin{align*}
        \sum& b_{n-1}(n-1)z^{n-2}=\sum -a_{n-2}z^{n-2}\\
        &\implies b_{n-1}=-a_{n-2}/(n-1)\\
        &\implies\sum a_nnz^{n-1}=\sum -a_{n-2}/(n-1)z^{n-1}\\
        &\implies a_n=-a_{n-2}/(n(n-1))\\
      \end{align*}
      Then we solve for $b_n$
      \begin{align*}
        \sum& a_{n-1}(n-1)z^{n-2}=\sum b_{n-2}z^{n-2}\\
        &\implies a_{n-1}=b_{n-2}/(n-1)\\
        &\implies b_n=-b_{n-2}/(n(n-1))
      \end{align*}
      Now we will assume that $s(0)=0,c(0)=1$. I claim that $s(z)=\sin(z)$, meaning even powers have a coefficient of 0. We have $a_0=0$
      \begin{align*}
        a_2&=-a_0/(2\cdot1)=0\\
        \letv a_{2n}=0\implies a_{2n+2}&=-a_{2n}/(2n(2n-1))=0
      \end{align*}
      And the odd terms will give us $a_{2n+1}=(-1)^n/(2n+1)!$. We have $b_0=1$ so
      \begin{align*}
        a_1&=b_0=1\\
        a_3&=-a_1/(3\cdot2)=(-1)/3!\\
        \letv a_{2n+1}&=(-1)^n/(2n+1)!\\
        \implies a_{2(n+1)+1}&=\frac{-(-1)^n}{(2n+1)!(2n+3)(2n+2)}\\
        &=\frac{(-1)^{n+1}}{(2(n+1)+1)!}
      \end{align*}
      Which gives the sum:
      \[\sum \frac{(-1)^nz^{2n+1}}{(2n+1)!}=\sin(z)\]
      so $s'(z)=\cos(z)=c(z)$ thus $c^2(z)+s^2(z)=1$
  \end{enumerate}
  \section{5.10 Problems: 1, 5, 15}
  \begin{enumerate}
    \item[1]
      \begin{enumerate}[label=(\roman*)]
        \item\[e^i=\cos(1)+i\sin(1)\]
        \item\[e^{2+i\pi}=e^2\cos(\pi)+e^2\sin(\pi)\]
        \item\[e^{-2-i\pi}=e^{-2}\cos(-\pi)+e^{-2}\sin(-\pi)\]
      \end{enumerate}
    \item[5]
      \begin{align*}
        \cos(\theta+\phi)+i\sin(\theta+\phi)&=e^{i(\theta+\phi)}\\
        &=e^{i\theta}e^{i\phi}\\
        &=(\cos(\theta)+i\sin(\theta))(\cos(\phi)+i\sin(\phi))\\
        &=\cos(\theta)\cos(\phi)-\sin(\theta)\sin(\phi)+i(\cos(\theta)\sin(\phi)+\cos(\phi)\sin(\theta))
      \end{align*}
      For the next relation:
      \begin{align*}
        \frac{1}{\cos(\theta)+i\sin(\theta)}&=e^{-i\theta}\\
        &=\cos(-\theta)+i\sin(-\theta)\\
        &=\cos(\theta)-i\sin(\theta)
      \end{align*}
    \item[15]
      \begin{enumerate}[label=(\roman*)]
        \item
          \begin{align*}
            \sum_{m=0}^n\cos(mx)+i\sin(mx)&=\sum_{m=0}^ne^{imx}\\
            &=\sum_{m=0}^n\left(e^{ix}\right)^m\\
            &=\frac{1-e^{ix(n+1)}}{1-e^{ix}}\\
            &=\frac{(1-e^{ix(n+1)})(1-e^{-ix})}{(1-e^{ix})(1-e^{-ix})}\\
            &=\frac{1-e^{-ix}-e^{ixn+ix}+e^{ixn}}{2-2\cos(x)}\\
            &=\frac{1-\cos(x)+i\sin(x)-\cos(xn+x)-i\sin(xn+x)+\cos(xn)+i\sin(xn)}{2-2\cos(x)}\\
          \end{align*}
          This gives us
          \[\sum_{m=0}^n\cos(mx)=\frac{1-\cos(x)-\cos(nx+x)+\cos(nx)}{2-2\cos(x)}\]
        \item
          From the previous problem we have:
          \[\sum_{m=0}^n\sin(mx)=\frac{\sin(x)+\sin(nx)-\sin(nx+x)}{2-2\cos(x)}\]
        \item
          \begin{align*}
            \sum_{m=1}^n\cos((2m-1)x)+i\sin((2m-1)x)&=\sum_{m=0}^ne^{2ixm}e^{-ix}\\
            &=e^{-ix}\sum_{m=0}^n\left(e^{i(2x)}\right)^m\\
            &=e^{-ix}\frac{1-e^{-i2x}-e^{i2xn+i2x}+e^{i2xn}}{2-2\cos(2x)}+e^{-ix}\\
            &=\frac{e^{-ix}-e^{-i3x}-e^{i2xn+ix}+e^{i2xn-ix}}{2-2\cos(2x)}+e^{-ix}\\
            \sum_{m=0}^n\cos((2m-1)x)&=\frac{\cos(x)-\cos(3x)-\cos(2xn+x)+\cos(2xn-x)}{2-2\cos(2x)}+\cos(x)
          \end{align*}
        \item
          By the previous problem:
          \begin{align*}
            \sum_{m=1}^n\sin((2m-1)x)=\frac{-\sin(x)+\sin(3x)-\sin(2xn+x)+\sin(2xn-x)}{2-2\cos(2x)}-\sin(x)
          \end{align*}
        \item
          The problem uses $(-1)^n$ as a coefficient. This is constant unlike $(-1)^m$ so we can just multiply it by our previous answer.
          \begin{align*}
            \sum_{m=0}^n(-1)^n(\cos(mx)+i\sin(mx))&=(-1)^n\sum_{m=0}^ne^{imx}\\
            \implies\sum_{m=0}^n(-1)^n\sin(mx)&=(-1)^n\frac{\sin(x)+\sin(nx)-\sin(nx+x)}{2-2\cos(x)}
          \end{align*}
        \item
          \begin{align*}
            \sum_{m=0}^n\cos(\theta+m\phi)+i\sin(\theta+m\phi)&=e^{i\theta} \sum_{m=0}^ne^{im\phi}\\
            &=\frac{e^{i\theta}-e^{-i\phi+i\theta}-e^{in\phi+i\phi+i\theta}+e^{in\phi+i\theta}}{2-2\cos(\phi)}\\
            \sum_{m=0}^n\cos(\theta+m\phi)&=\frac{\cos(\theta)-\cos(\theta-\phi)-\cos(n\phi+\phi+\theta)+\cos(n\phi+\theta)}{2-2\cos(\phi)}
          \end{align*}
        \item
          From the previous problem:
          \begin{align*}
            \sum_{m=0}^n\sin(\theta+m\phi)&=\frac{\sin(\theta)-\sin(\theta-\phi)-\sin(n\phi+\phi+\theta)+\sin(n\phi+\theta)}{2-2\cos(\phi)}
          \end{align*}
      \end{enumerate}
  \end{enumerate}
  \section{6.14 Problems: 1, 2, 4, 5, 12, 13}
  \begin{enumerate}
    \item[1]
      We have $f(a,b)=a$ and $\gamma(t)=it$ for $t\in[0,1]$
      \begin{align*}
        f(\gamma(t))=0\\
        \int_\gamma re(z)dz=\int_0^1f(\gamma(t))\gamma'(t)dt=0\\
      \end{align*}
      \begin{align*}
        \sigma_1(t)=t&\implies \sigma_1'(t)=1\\
        \sigma_2(t)=1-t+it&\implies \sigma_2'(t)=-1+i\\
        \int_{\sigma_1} f(\sigma_1(t))\sigma_1'(t)dt+\int_{\sigma_2}f(\sigma_2(t))\sigma_2'(t)dt&=\int_0^1 tdt+\int_0^1(1-t)(i-1)dt\\
        &=1/2+\int_0^1i-it-1+tdt\\
        &=1/2+i(1)-i(1)^2/2-(1)+(1)^2/2\\
        &=i/2
      \end{align*}
    \item[2]
      We have the contour $\gamma(t)=it$ for $t\in[-1,1]$. We will find the derivative $\gamma'(t)=i$. If we find the magnitude of the contour then $|\gamma(t)|=t$ so we can evaluate the integral
      \[\int_{\gamma}|z|dz=\int_{-1}^1itdt=\frac{i(1)^2}{2}-\frac{i(-1)^2}{2}=0\]
      We can take the derivative of the second contour to get $\sigma'(t)=ie^{it}$. We also have $|\sigma(t)|=1$ so we can evaluate the integral
      \[\int_{\sigma}|\sigma(t)|\sigma'(t)dt=\int_{-\pi/2}^{\pi/2}ie^{it}dt=e^{i(\pi/2)}-e^{i(-\pi/2)}=i-(-i)=2i\]
    \item[4]
      We must use the basic formula. Plugging in the contour we get $\frac{1}{z_0+re^{it}-z_0}=e^{-it}/r$. And the derivative is $\gamma'(t)=ire^{it}$ Giving us the integral:
      \[\int_\gamma \frac{ire^{it}}{re^{it}}dt=\int_{0}^{2n\pi}idt=2in\pi\]
    \item[5]
      For all problems we have $\gamma(t)=e^{it}$ and $\gamma'(t)=ie^{it}$.
      \begin{enumerate}[label=(\roman*)]
        \item
          We can solve the integral to get
          \[\int_{1}^{-1} 1/z^2dz=-1/(-1)-(-1/(1))=2\]
        \item
          Plug in the contour
          \[f(\gamma(t))=e^{-it}\]
          Then solve the integral
          \[\int_{0}^{\pi}e^{-it}\cdot ie^{it}dt=i\pi\]
        \item
          We will simply take the integral to get
          \[\int_1^{-1}\cos(z)dz=\sin(-1)-\sin(1)=-2\sin(1)\]
        \item
          Simply take the integral
          \[\int_1^{-1}\sinh(z)dz=\cosh(-1)-\cosh(1)=-2\cosh(1)\]
        \item
          We will need a u-substitution
          \begin{align*}
            \int_1^{-1}\sin(z)/\cos(z)dz\\
            u=\cos(z)\implies du=-\sin(z)dz\\
            \int_1^{-1}\frac{1}{u}du=\ln(\cos(-1))-\ln(\cos(1))=-2\ln(\cos(1))
          \end{align*}
        \item
          We will plug in the contour
          \[f(e^{it})=e^{(e^{it})^3}=e^{e^{3it}}\]
          Now we integrate
          \begin{align*}
            \int_1^{-1}\sum_{n=0}^\infty \frac{z^{3n}}{n!}dz&=\sum_{n=0}^\infty\int_1^{-1} \frac{z^{3n}}{n!}\\
            &=\sum_{n=1}^\infty \frac{(-1)^{3n-1}}{n!(3n-1)}-\sum_{n=1}^\infty \frac{(1)^{3n-1}}{n!(3n-1)}
          \end{align*}
      \end{enumerate}
    \item[12]
      (i) First we must find the length of the contour. Because it is a circle we can just find the perimeter and get $2\pi r$. Suppose we have some $M\st|f(z)|\leq M$. Given some $\epsilon>0$ we will then let $\delta=\frac{\epsilon}{2M\pi}$
      \begin{align*}
        |r|<\delta\implies\\
        \left|\int_{c_r}f(z)dz\right|\leq|2Mr\pi|<\epsilon
      \end{align*}
      (ii) We will let the
  \end{enumerate}
\end{document}
