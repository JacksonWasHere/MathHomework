\documentclass{article}

\usepackage{jacksonmath}
\usepackage[english]{babel}
\usepackage[utf8]{inputenc}
\usepackage[final]{pdfpages}

\begin{document}
  \section{4.7 Problems: 11, 12, 15}
  \begin{enumerate}
    \item[11]
      We will consider the equation:
      \begin{align*}
        &f(z)=\frac{xy^2(x+iy)}{x^2+x^4}\\
        &f(0)=0
      \end{align*}
      Let us take the limit when $z\to0$ along any line $z=(a+bi)t$
      \begin{align*}
        \lim_{t\to0}\frac{f((a+ib)t)}{(a+ib)t}&=\frac{(at)(bt)^2(a+ib)t}{((at)^2+(at)^4)(a+ib)t}\\
        &=\frac{ab^2t^3}{a^2t^2+a^4t^4}\\
        &=\frac{ab^2t^1}{a^2+a^4t^2}\\
        &=0
      \end{align*}
      So the limit goes to zero when $z\to0$ along a straight path. But what if we take a different path to 0. We will instead let $z(t)=t^2+it$ as $z\to0$.
      \begin{align*}
        \lim_{t\to0}\frac{f(t^2+it)}{t^2+it}&=\frac{t^2t^2(t^2+it)}{(t^4+t^8)(t^2+it)}\\
        &=\frac{t^4}{t^4+t^8}\\
        &=\frac{1}{1+t^4}\\
        &=1
      \end{align*}
    \item[12]
      \begin{enumerate}[label=(\roman*)]
        \item
          First we will find $f'$
          \[f'(z)=(z^2)'=2z\]
          Then we will plug in the path
          \[f'(\gamma(t))=2(t^3+it^4)\]
          We can easily find the derivative of the path
          \[\gamma'(t)=3t^2+4it^3\]
          Now we want to find the derivative of the compostion
          \begin{align*}
            (f\gamma)'(t)&=\left((t^3+it^4)^2\right)'\\
            &=\left(t^6-t^8+2it^7\right)'\\
            &=-8t^7+14it^6+6t^5\\
            f'(\gamma(t))\gamma'(t)&=2(t^3+it^4)(3t^2+4it^3)\\
            &=2(3t^5+7it^6-4t^7)\\
            &=-8t^7+14it^6+6t^5
          \end{align*}
          So $f'(\gamma(t))\gamma'(t)=(f\gamma)'(t)$.
        \item
          First we will find $f'$
          \[f'(z)=(1/z)'=-1/z^2\]
          Then we will plug in the path
          \[f'(\gamma(t))=-1/(\cos(t)+i\sin(t))^2=-e^{-2it}\]
          We can easily find the derivative of the path
          \[\gamma'(t)=-\sin(t)+i\cos(t)=ie^{it}\]
          Now we want to find the derivative of the compostion
          \begin{align*}
            (f\gamma)'&=\left(\frac{1}{\cos(t)+i\sin(t)}\right)'\\
            &=\left(e^{-it}\right)'\\
            &=-ie^{-it}\tag{1}\\
            f'(\gamma(t))\gamma'(t)&=-e^{-2it}ie^{it}\\
            &=-ie^{-it}\tag{2}
          \end{align*}
          (1)=(2) so $f'(\gamma(t))\gamma'(t)=(f\gamma)'(t)$.
        \item
          First we will find $f'$
          \[f'(z)=\left(\sum z^n\right)'=\sum nz^{n-1}\]
          Then we will plug in the path
          \[f'(\gamma(t))=\sum n(t+it^2)^{n-1}\]
          We can easily find the derivative of the path
          \[\gamma'(t)=1+2it\]
          Now we want to find the derivative of the compostion
          \begin{align*}
            (f\gamma)'&=\left(\sum (t+it^2)^n\right)'\\
            &=\sum n(t+it^2)^{n-1}(1+2it)\\
            f'(\gamma(t))\gamma'(t)&=(1+2it)\sum n(t+it^2)^{n-1}\\
            &=\sum n(t+it^2)^{n-1}(1+2it)
          \end{align*}
          So $f'(\gamma(t))\gamma'(t)=(f\gamma)'(t)$.
      \end{enumerate}
    \item[15]
      Let's consider the series $s(z)=\sum a_nz^n, c(z)=\sum b_n,z^n$. Let's take the derivative of $s(z)$ and $c(z)$:
      \[s'(z)=\sum a_nnz^{n-1}\]
      \[c'(z)=\sum b_nnz^{n-1}\]
      We can then work out the coefficients because we know $s'(z)=c(z)$ and $c'(z)=-s(z)$:
      \begin{align*}
        \sum a_nnz^{n-1}&=\sum b_{n-1}z^{n-1}\tag{1}\\
        \sum b_nnz^{n-1}&=\sum -a_{n-1}z^{n-1}\tag{2}\\
      \end{align*}
      If we combine (1) and (2) to solve for $a_n$
      \begin{align*}
        \sum& b_{n-1}(n-1)z^{n-2}=\sum -a_{n-2}z^{n-2}\\
        &\implies b_{n-1}=-a_{n-2}/(n-1)\\
        &\implies\sum a_nnz^{n-1}=\sum -a_{n-2}/(n-1)z^{n-1}\\
        &\implies a_n=-a_{n-2}/(n(n-1))\\
      \end{align*}
      Then we solve for $b_n$
      \begin{align*}
        \sum& a_{n-1}(n-1)z^{n-2}=\sum b_{n-2}z^{n-2}\\
        &\implies a_{n-1}=b_{n-2}/(n-1)\\
        &\implies b_n=-b_{n-2}/(n(n-1))
      \end{align*}
      Now we will assume that $s(0)=0,c(0)=1$. I claim that $s(z)=\sin(z)$, meaning even powers have a coefficient of 0. We have $a_0=0$
      \begin{align*}
        a_2&=-a_0/(2\cdot1)=0\\
        \letv a_{2n}=0\implies a_{2n+2}&=-a_{2n}/(2n(2n-1))=0
      \end{align*}
      And the odd terms will give us $a_{2n+1}=(-1)^n/(2n+1)!$. We have $b_0=1$ so
      \begin{align*}
        a_1&=b_0=1\\
        a_3&=-a_1/(3\cdot2)=(-1)/3!\\
        \letv a_{2n+1}&=(-1)^n/(2n+1)!\\
        \implies a_{2(n+1)+1}&=\frac{-(-1)^n}{(2n+1)!(2n+3)(2n+2)}\\
        &=\frac{(-1)^{n+1}}{(2(n+1)+1)!}
      \end{align*}
      Which gives the sum:
      \[\sum \frac{(-1)^nz^{2n+1}}{(2n+1)!}=\sin(z)\]
      so $s'(z)=\cos(z)=c(z)$ thus $c^2(z)+s^2(z)=1$
  \end{enumerate}
  \section{5.10 Problems: 1, 5, 15}
  \begin{enumerate}
    \item[1]
      \begin{enumerate}[label=(\roman*)]
        \item\[e^i=\cos(1)+i\sin(1)\]
        \item\[e^{2+i\pi}=e^2\cos(\pi)+e^2\sin(\pi)\]
        \item\[e^{-2-i\pi}=e^{-2}\cos(-\pi)+e^{-2}\sin(-\pi)\]
      \end{enumerate}
    \item[5]
      \begin{align*}
        \cos(\theta+\phi)+i\sin(\theta+\phi)&=e^{i(\theta+\phi)}\\
        &=e^{i\theta}e^{i\phi}\\
        &=(\cos(\theta)+i\sin(\theta))(\cos(\phi)+i\sin(\phi))\\
        &=\cos(\theta)\cos(\phi)-\sin(\theta)\sin(\phi)+i(\cos(\theta)\sin(\phi)+\cos(\phi)\sin(\theta))
      \end{align*}
      For the next relation:
      \begin{align*}
        \frac{1}{\cos(\theta)+i\sin(\theta)}&=e^{-i\theta}\\
        &=\cos(-\theta)+i\sin(-\theta)\\
        &=\cos(\theta)-i\sin(\theta)
      \end{align*}
    \item[15]
      \begin{enumerate}[label=(\roman*)]
        \item
          \begin{align*}
            \sum_{m=0}^n\cos(mx)+i\sin(mx)&=\sum_{m=0}^ne^{imx}\\
            &=\sum_{m=0}^n\left(e^{ix}\right)^m\\
            &=\frac{1-e^{ix(n+1)}}{1-e^{ix}}\\
            &=\frac{1-\cos(x(n+1))-i\sin(x(n+1))}{1-\cos(x)-i\sin(x)}
          \end{align*}
      \end{enumerate}
  \end{enumerate}
\end{document}
