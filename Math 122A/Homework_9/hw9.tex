\documentclass{article}


\usepackage[english]{babel}
\usepackage[utf8]{inputenc}
\usepackage[final]{pdfpages}
\usepackage{jacksonmath}

\begin{document}
  \section{Worksheet Problems:}
  \begin{enumerate}
    \item
      Let us consider the function $g(z)=f(z)f(-z)$. If $z$ were to tend to zero from the positive direction then we would get
      \begin{align*}
        \lim_{z\to0^+}g(z)&=\lim_{z\to0^+}f(z)f(-z)\\
        &\leq\lim_{z\to0^+}2\cdot3\\
        &=6
      \end{align*}
      Obviously $g(0)=f(0)^2$ thus $|f(0)|\leq\sqrt{6}$
    \item
      Consider some $z=x+iy$, let us find the maximum and minimum of the function.
      \[e^z=e^{x+iy}=e^xe^{iy}\]
      We can then see that the modulus of the function is
      \[\left|e^z\right|=\abs*{e^xe^{iy}}=e^x\]
      This means that the maximum of the modulus is the rightmost point in the set. This must be on the boundary. If we had a right most point inside the boundary then we could simply move right until we reach the boundary. The minimum is when we have the greatest negative value of $x$. This must also be on the boundary by the same logic.
    \item
      First we shall factor to get
      \[f(z)=z(z-1)\]
      The modulus:
      \[\abs*{z(z-1)}=\abs{z}\abs{z-1}\]
      This would mean that the maximum is when both moduli are at a maximum. At all point on the boundary of the disk $|z|$ is a maximum. But only at the point farthest from 1 for $|z-1|$ to be at a maximum. Thus the point $z_0=-1$ and $|f(-1)|=|-1||-1-1|=2$. The minimum is 0 and this is at points $z_0=0,1$.
    \item
      Assume that the polynomial $p(z)$ is non constant and differentiable. Consider a closed disk $D$ with radius $r$ centered at the origin. By the maximum modulus theorem we know that if $|p(z_0)|\geq|p(z)|$ for all $z\in D$ then $z_0$ must be on the boundary. If we increase $r$ then $|z_0|=r$ will also increase. This means that as $|z_0|\to\infty$ then $|p(z_0)|\to\infty$. Hence we can choose some $r$ such that
      \[\forall z\in\C-D,|p(z)|>\max_D|p|\]
      The minimum then must be insides of $D$, which by the minimum modulus theorem means they are zeros.
    \item[17]
      Consider the function $f(z)=1+z$, in a disk $D$ around the origin with radius $a$. We know that the maximum must be on the boundary because it is not constant. The modulus is
      \[|f(z)|=\sqrt{(1+x)^2+y^2}\]
      It must be that
      \[y^2=a^2-x^2\]
      so
      \[\sqrt{(1+x)^2+a^2-x^2}=\sqrt{1+2x+a^2}\]
      which is clearly a maximum when $x=a$.
  \end{enumerate}
  \section{Chapter 11 Problems:}
  \begin{enumerate}
    \item
      \begin{enumerate}[label=(\roman*)]
        \item$f(z)=(z-3)^{-1}$\\
          Let us first rewrite
          \[\frac{1}{-3}\cdot\frac{1}{1-z/3}\]
          By the binomial expansion we get the result
          \[\frac{1}{3}\sum_{n=0}^\infty\left(\frac{x}{3}\right)^n\]
          in the annulus $0<|z|<3$
        \item$f(z)=(z-a)^{-k}$\\
          Let us rewrite this as
          \[\frac{1}{(-a)^k(1-z/a)^k}\]
          we can then use the binomial series
          \begin{align*}
            \frac{1}{(-a)^k}(1-z/a)^{-k}=
            (-1)^{k}\sum_{n=0}^\infty
              \binom{k+n-1}{n}\left(\frac{z^n}{a^{k+n}}\right)
          \end{align*}
          Giving us the series
          in the annulus $0<|z|<a$
        \item
          We will partial fraction
          \[\frac{1}{z(1-z)}=\frac{1}{z}+\frac{1}{1-z}\]
          Then we can use the binomial expansion
          \[\frac{1}{z}+\sum_{n=0}^\infty z^n\]
          The annulus is $0<|z|<1$
        \item
          Again we can use partial fractions.
          \[\frac{1}{(z-a)(z-b)}=\frac{1}{z(a-b)(1-a/z)}+\frac{1}{(ab-b^2)(1-z/b)}\]
          Let us note that either $|a|\leq|b|$ or $|b|\leq|a|$, both cases would be identical so we will assume $|a|\leq|b|$. Then we will use the binomial expansion.
          \begin{align*}
            \frac{1}{z(a-b)(1-a/z)}
            &=\frac{1}{(a-b)}\sum_{n=0}^\infty\left(\frac{a^n}{z^{n+1}}\right)\\
            \frac{1}{(ab-b^2)(1-z/b)}
            &=\frac{1}{ab-b^2}\sum_{n=0}^\infty\left(\frac{z}{b}\right)^n
            f(z)&=\frac{1}{(a-b)}\sum_{n=0}^\infty
              \left(\frac{a^n}{z^{n+1}}\right)
            +\frac{1}{ab-b^2}\sum_{n=0}^\infty
              \left(\frac{z}{b}\right)^n
          \end{align*}
          and this is valid in the annulus $|a|<|z|<|b|$
        \item
          It is known that $e^{1/z}$ has the power series
          \[\sum_{n=0}^\infty\frac{1}{n!}z^{-n}\]
          so we can easily obtain
          \[\sum_{n=0}^\infty\frac{1}{n!}z^{3-n}=z^3+z^2+\frac{1}{2}z+\sum_{n=0}^\infty\frac{1}{(n+3)!}z^{-n}\]
          Which converges on the annulus $z\neq0$
        \item
          We can put $z+1/z$ into the series for $e^z$ and use the binomial expansion.
          \begin{align*}
            \sum_{k=0}^\infty\frac{(z+1/z)^k}{k!}
            &=\sum_{k=0}^\infty\frac{1}{k!}\left(\sum_{n=0}^k
              \binom{k}{n}z^{2n-k}\right)\\
          \end{align*}
          We can now try to find the coefficients of negative powers with this series. Consider when $2n-k=a$ and try to find the coefficient for $z^{a}$.
          \begin{align*}
            2n-k=a&\implies k=2n-a\\
            \frac{1}{k!}\binom{k}{n}z^{2n-k}&\implies\frac{1}{(2n-a)!}\binom{2n-a}{n}z^{a}\\
            &\implies e^{z+1/z}\sum_{a=-\infty}^\infty\sum_{n=0}^\infty\frac{z^a}{(2n-a)!}\binom{2n-a}{n}
          \end{align*}
          and $0<|z|$
        \item
          We will use the regular series for $\cos(z)$ but use $1/z$.
          \[\sum_{n=0}^\infty\frac{(-1)^nz^{-2n}}{(2n)!}\]
          The annulus would be $0<|z|$
        \item
          Similarly to $e^{1/z}$ we will put $z^{-5}$ into the expansion of $e^z$,
          \[\sum_{n=0}^\infty\frac{z^{-5n}}{n!}\]
          This also has annulus $0<|z|$
      \end{enumerate}
    \item
      \begin{enumerate}[label=(\roman*)]
        \item
          We will first use partial fractions
          \[\frac{1}{(z-1)^2(z-2)}=\frac{-3z}{(z-1)^2}+\frac{4}{(z-1)^2}+\frac{2}{z-2}\]
          From 11.7.1(ii)
          \begin{align*}
            \frac{-3z}{(z-1)^2}&=-3\sum_{n=0}^\infty (n+1)z^{n+1}\\
            \frac{4}{(z-1)^2}&=4\sum_{n=0}^\infty (n+1)z^n\\
            \frac{2}{z-2}&=-\sum_{n=0}^\infty\frac{z^n}{2^{n+1}}\\
            f(z)&=\sum_{n=0}^\infty (n+4-2^{-n-1})z^n
          \end{align*}
        \item
          Similarly to (i)
          \[\frac{1}{(z-1)^2(z-2)}=\frac{-3z}{(z-1)^2}+\frac{4}{(z-1)^2}+\frac{2}{z-2}\]
          The first two series must change slightly
          \begin{align*}
            \frac{-3z}{z^2(1-1/z)^2}
            &=-3\sum_{n=0}^\infty(n+1)\frac{2^n}{z^{n+1}}\\
            \frac{4}{z^2(1-1/z)^2}
            &=4\sum_{n=0}^\infty(n+1)\frac{2^n}{z^{n+2}}\\
            \frac{2}{z-2}
            &=-\sum_{n=0}^\infty\frac{z^n}{2^{n+1}}\\
            f(z)&=\sum_{n=0}^\infty(n2^{n+1}-3(n+1)2^n)z^{-n-1}-\sum_{n=0}^\infty\frac{z^n}{2^{n+1}}
          \end{align*}
        \item
          Similarly to (i) and (ii)
          \[\frac{1}{(z-1)^2(z-2)}=\frac{-3z}{(z-1)^2}+\frac{4}{(z-1)^2}+\frac{2}{z-2}\]
          The first two series must change slightly
          \begin{align*}
            \frac{-3z}{z^2(1-1/z)^2}
            &=-3\sum_{n=0}^\infty(n+1)\frac{2^n}{z^{n+1}}\\
            \frac{4}{z^2(1-1/z)^2}
            &=4\sum_{n=0}^\infty(n+1)\frac{2^n}{z^{n+2}}\\
            \frac{2}{2(1-2/z)}
            &=\sum_{n=0}^\infty\frac{2^{n+1}}{z^{n+1}}\\
            f(z)&=\sum_{n=0}^\infty((n+1)2^{n+1}-3(n+1)2^n)z^{-n-1}
          \end{align*}
        \item
          We can use the series for $e^z$.
          \begin{align*}
            e^{-z^{-2}}=\sum_{n=0}^\infty\frac{(-1)^nz^{-2n}}{n!}
          \end{align*}
        \item
          Consider the disk $0<|z-1|<1+(1-\sqrt5)/2$. We can factor to get
          \[\frac{1}{(1-z-z^2)}=\frac{1}{(\frac{\sqrt{5}-1}{2}-z)(\frac{\sqrt{5}+1}{2}+z)}\]
          We can then split this with partial fractioning
          \[\frac{1/\sqrt{5}}{(\frac{\sqrt{5}-1}{2}-z)}
           +\frac{1/\sqrt{5}}{(\frac{\sqrt{5}+1}{2}+z)}\]
          We can try to find a series of $z-1$ (whatever that means) for each of these fractions.
        \item
          Using partial fractions we get
          \begin{align*}
            \frac{1}{(1+z^2)}=\frac{1}{(z-i)(z+i)}=\frac{i/2}{(i+z)}+\frac{i/2}{(i-z)}
          \end{align*}
          We can find the series for each of these
          \begin{align*}
            \frac{1/2}{(1-zi)}&=\frac{1}{2}\sum_{n=0}^\infty i^nz^n\\
            \frac{1/2}{1+iz}&=\frac{1}{2}\sum_{n=0}^\infty i^{-n}z^n\\
            e^z&=\sum_{n=0}^\infty\frac{z^n}{n!}
          \end{align*}
          Now we must find
          \begin{align*}
            e^z/(1+z^2)&=\frac{1}{2}\left(\sum_{n=0}^\infty\frac{z^n}{n!}\right)
              \left(\sum_{n=0}^\infty (i^n+i^{-n})z^n\right)\\
            &=\frac{1}{2}\left(\sum_{n=0}^\infty z^n
              \sum_{k=0}^n\frac{i^k+i^{-k}}{(n-k)!}\right)
          \end{align*}
      \end{enumerate}
  \end{enumerate}
\end{document}
