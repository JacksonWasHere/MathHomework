\documentclass{article}

\usepackage{jacksonmath}
\usepackage[english]{babel}
\usepackage[utf8]{inputenc}
\usepackage[final]{pdfpages}

\begin{document}
  \begin{enumerate}
    \item[11]
      Suppose that Cauchy's estimate is an equality. Let's create a disk around the origin of radius $r>0$. Using Cauchy's estimate we get
      \begin{align*}
        |f^{(0)}(0)|=|f(0)|&=M
      \end{align*}
      This means that the origin is a maximum. Suppose we let $f(z)=Kz^n$ then in the disk $|f(z)|=|K||z^n|\leq M=|f(0)|$. But this gives us
      \[|K|\leq\frac{|f(0)|}{|z^n|}\]
      We can let the disk be arbitrarily large as well as $z$ thus $|K|\leq\frac{|f(0)|}{|z^n|}\to0$. So $f(z)=0$\qed
    \item[12]
      For the domain $D$ we have a fixed center $z_0$ and a radius $r$. $\partial D$ can be written as the path $\partial D(t)=z_0+re^{it}$ with $t\in[0,2\pi]$. If we integrate $f(\partial D(t))$ we get the sum of all $f(z)$ along the path so to get the average we can divide by the length of the path giving us:
      \[
        \frac{1}{2\pi}\int_0^{2\pi}f(\partial D(t))dt=\frac{1}{2\pi}\int_0^{2\pi}f(z_0+re^{it})dt
      \]
      We can then do a substitution with $z=z_0+re^{it}$ and $dz=ire^{it}dt$.
      \begin{align*}
        \frac{1}{2\pi}\int_0^{2\pi}\frac{f(z)}{ire^{it}}dt&=\frac{1}{2i\pi}\int_0^{2\pi}\frac{f(z)}{re^{it}}dz\\
        &=\frac{1}{2i\pi}\int_{\partial D}\frac{f(z)}{z-z_0}dz
      \end{align*}
      by Cauchy's integral formula we can see
      \[\frac{1}{2i\pi}\int_{\partial D}\frac{f(z)}{z-z_0}dz=f(z_0)\]\qed
    \item[13]
      Suppose we have a domain $D$ with radius $r>0$. The maximum would be $|f(z)|\leq Kr^c$. Then using Cauchy's estimate for $n=c$:
      \begin{align*}
        |f^{(c)}(z)|\leq\frac{Kr^c(c)!}{r^c}=Kc!
      \end{align*}
      This means that $f^{(c)}(z)=w$ and $w$ is constant. Let's find the antiderivative of $f^{(c)}(z)$, $c$ times:
      \begin{align*}
        f^{(c-1)}(z)&=wz+l_0\text{(l is a constant of integer)}\\
        f^{(c-2)}(z_0)&=\frac{wz^2}{2!}+l_0z+l_1\\
        f^{(c-(c-1))}(z_0)&=\frac{wz^{c-1}}{(c-1)!}+l_0\frac{z^{c-2}}{(c-2)!}+l_1\frac{z^{c-3}}{(c-3)!}+...\\
        f(z_0)&=\frac{wz^{c}}{(c)!}+l_0\frac{z^{c-1}}{(c-1)!}+l_1\frac{z^{c-2}}{(c-2)!}+...
      \end{align*}
      This gives us a polynomial of degree $\leq c$\qed
    \item[14]
      Let's consider the taylor expansion of $f,g$ centered at 0
      \[
        f(z)=\sum_{n=0}^\infty\frac{f^{(n)}(0)z^n}{n!}
      \]\[
        g(z)=\sum_{n=0}^\infty\frac{g^{(n)}(0)z^n}{n!}
      \]
      We also know that
      \[f^{(n)}(0)=\frac{n!}{2i\pi}\left|\int_{C_r}\frac{f(z)}{(z-z_0)^{n+1}}dz\right|\]
      if we choose a small enough $r$ that would mean
      \begin{align*}
        \frac{n!}{2i\pi}\left|\int_{C_r}\frac{f(z)}{(z-z_0)^{n+1}}dz\right|=\frac{n!}{2i\pi}\left|\int_{C_r}\frac{g(z)}{(z-z_0)^{n+1}}dz\right|
      \end{align*}
      Thus:
      \begin{align*}
        f^{(n)}(0)&=g^{(n)}(0)\\
        \implies f(z)&=\sum_{n=0}^\infty\frac{f^{(n)}(0)z^n}{n!}\\
        &=\sum_{n=0}^\infty\frac{g^{(n)}(0)z^n}{n!}\\
        &=g(z)
      \end{align*}\qed
    \item[15]
      Suppose that $f(z)=\sum a_n(z-z_0)^n$ in a disk D center $z_0$ radius $R$. Assume $0\leq r<R$.
      \begin{align*}
        \frac{1}{2\pi}\int_{0}^{2\pi}\left|f(z_0+re^{i\theta})\right|^2d\theta
        &=\frac{1}{2\pi}\int_{0}^{2\pi}
          f(z_0+re^{i\theta})
          \overline{f(z_0+re^{i\theta})}d\theta\\
        &=\frac{1}{2\pi}\int_{0}^{2\pi}
          \left(\sum_{n=0}^\infty a_nr^ne^{in\theta}\right)
          \left(\sum_{n=0}^\infty \con{a_n}r^ne^{-in\theta}\right)d\theta\\
        &=\frac{1}{2\pi}\int_{0}^{2\pi}
          \sum_{n=0}^{\infty}\sum_{k=0}^{n}
          (a_kr^ke^{ik\theta})(\con{a_{n-k}}r^{n-k}e^{-i(n-k)\theta})d\theta\\
        &=\frac{1}{2\pi}
          \sum_{n=0}^{\infty}\sum_{k=0}^{n}\int_{0}^{2\pi}
          (a_{k}\con{a_{n-k}}r^ne^{i(2k-n)\theta})d\theta\\
      \end{align*}
      Let us consider different terms in this sequence. Suppose that $2k-n\neq0$, so the integral would be:
      \begin{align*}
        \left[a_{k}\con{a_{n-k}}r^n
        \frac{e^{i(2k-n)\theta}}{i(2k-n)}\right]_{0}^{2\pi}=
        a_{k}\con{a_{n-k}}r^n
        \frac{e^{i(2k-n)2\pi}-e^{i(2k-n)}}{i(2k-n)}=0\\
      \end{align*}
      But if instead $2k-n=0$
      \begin{align*}
        \int_{0}^{2\pi}a_{k}\con{a_{n-k}}r^n
        e^{i(2k-n)\theta}d\theta&=
        \int_{0}^{2\pi}|a_k|^2r^nd\theta\\
        &=2\pi|a_k|^2r^n
      \end{align*}
      Now let us see when each case occurs. When $n$ is even then
      \[\sum_{k=0}^{n}\int_{0}^{2\pi}
        (a_{k}\con{a_{n-k}}r^ne^{i(2k-n)\theta})d\theta=2\pi|a_{n/2}|^2r^n\]
      if $n$ is odd then all terms become $0$.
      Thus the series is
      \[\sum_{n=0}^\infty|a_n|^2r^{2n}\]
      We know that $|f(z_0+re^{i\theta})|^2\leq\sup_\theta|f(z_0+re^{i\theta})|$, so clearly
      \begin{align*}
        \sum_{n=0}^\infty|a_n|^2r^{2n}
        &=\frac{1}{2\pi}\int_{0}^{2\pi}
          \left|f(z_0+re^{i\theta})\right|^2d\theta\\
        &\leq\frac{1}{2\pi}\int_{0}^{2\pi}
          \sup_\theta\left|f(z_0+re^{i\theta})\right|^2d\theta\\
        &=\sup_\theta\left|f(z_0+re^{i\theta})\right|^2\\
      \end{align*}
      Now let us assume that we have a local maximum $z_0=0$. We can create a disk around $z_0$ within the radius of convergence. We can see that $|f(0)|=|a_0|$ so
      \begin{align*}
        &\left|\sum a_nz^n\right|\leq|a_0|\\
        &\implies \sum_{n=0}^\infty|a_n|^2r^{2n}=|a_0|^2+\sum_{n=1}^\infty|a_n|^2r^{2n}\leq|a_0|^2\\
        &\implies
        \sum_{n=1}^\infty|a_n|^2r^{2n}\leq0
      \end{align*}
      Thus the function must be constant.
      \qed
    \item[19]
      Suppose that there exists some $z_0$ such that $f(z_0)=0$. Because $f(z)$ is differentiable in it's domain there is a Taylor expansion around $z_0$
      \begin{align*}
        \letv a_n&=\frac{f^{(n)}(z_0)}{n!}\\
        f(z)&=\sum_{n=0}^\infty a_n(z-z_0)^n
      \end{align*}
      Suppose $f^{(m)}(z_0)\neq0$ thus we have
      \begin{align*}
        f(z)=(z-z_0)^m\sum_{n=0}^\infty a_{n+m}(z-z_0)^n
      \end{align*}
      then we know that the function is not identically zero. But if it is NOT of finite order then $\forall n\in\N,f^{(n)}(z_0)=0$. This would mean that $a_n=0$ which gives the sequence
      \[f(z)=\sum_{n=0}^\infty 0\cdot z^n=0\]
      So $f$ is identically 0.\qed
  \end{enumerate}
\end{document}
