\documentclass{article}

\usepackage{jacksonmath}
\usepackage[english]{babel}
\usepackage[utf8]{inputenc}
\usepackage[final]{pdfpages}

\begin{document}
  \begin{enumerate}
    \item[11]
      Suppose that Cauchy's estimate is an equality. Let's create a disk around $z_0$ with radius $r>0$.
      Using Cauchy's integral formula we get:
      \begin{align*}
        |f^{(n)}(z_0)|&=\frac{n!}{2\pi i}\left|\int_{C_r}\frac{f(z)}{(z-z_0)^{n+1}}dz\right|\\
        &=\frac{n!}{2\pi i}\int_{C_r}\left|\frac{f(z)}{(z-z_0)^{n+1}}\right|dz\\
        &=\frac{Mn!}{r^{n}}
      \end{align*}
      The second equality only holds if $\frac{f(z)}{(z-z_0)^{n+1}}$ has a constant argument. We can let $z=z_0+re^{it}$ and $dz=ire^{it}dt$. t therefore must be constant. giving us
      \begin{align*}
        \frac{n!}{2\pi}\int_{C_r}\left|\frac{f(z_0+re^{it})}{(re^{it})^{n}}\right|dt
        &=\frac{n!}{2\pi}\int_{C_r}\frac{|f(z_0+re^{it})|}{r^{n}}dt\\
        &=\frac{n!|f(z_0+re^{it})|}{r^n}
      \end{align*}
    \item[12]
      For the domain $D$ we have a fixed center $z_0$ and a radius $r$. $\partial D$ can be written as the path $\partial D(t)=z_0+re^{it}$ with $t\in[0,2\pi]$. If we integrate $f(\partial D(t))$ we get the sum of all $f(z)$ along the path so to get the average we can divide by the length of the path giving us:
      \[
        \frac{1}{2\pi}\int_0^{2\pi}f(\partial D(t))dt=\frac{1}{2\pi}\int_0^{2\pi}f(z_0+re^{it})dt
      \]
      We can then do a substitution with $z=z_0+re^{it}$ and $dz=ire^{it}dt$.
      \begin{align*}
        \frac{1}{2\pi}\int_0^{2\pi}\frac{f(z)}{ire^{it}}dt&=\frac{1}{2i\pi}\int_0^{2\pi}\frac{f(z)}{re^{it}}dz\\
        &=\frac{1}{2i\pi}\int_{\partial D}\frac{f(z)}{z-z_0}dz
      \end{align*}
      by Cauchy's integral formula we can see
      \[\frac{1}{2i\pi}\int_{\partial D}\frac{f(z)}{z-z_0}dz=f(z_0)\]
    \item[13]
      Suppose we have a domain $D$ with radius $r>0$. The maximum would be $|f(z)|\leq Kr^c$. Then using Cauchy's estimate for $n=c$:
      \begin{align*}
        |f^{(c)}(z)|\leq\frac{Kr^c(c)!}{r^c}=Kc!
      \end{align*}
      This means that $f^{(c)}(z)=w$ and $w$ is constant. Let's find the antiderivative of $f^{(c)}(z)$, $c$ times:
      \begin{align*}
        f^{(c-1)}(z)&=wz+l_0\text{(l is a constant of integer)}\\
        f^{(c-2)}(z_0)&=\frac{wz^2}{2!}+l_0z+l_1\\
        f^{(c-(c-1))}(z_0)&=\frac{wz^{c-1}}{(c-1)!}+l_0\frac{z^{c-2}}{(c-2)!}+l_1\frac{z^{c-3}}{(c-3)!}+...\\
        f(z_0)&=\frac{wz^{c}}{(c)!}+l_0\frac{z^{c-1}}{(c-1)!}+l_1\frac{z^{c-2}}{(c-2)!}+...
      \end{align*}
      This gives us a polynomial of degree $\leq c$
    \item[14]
      Let's consider the taylor expansion of $f,g$ centered at 0
      \[
        f(z)=\sum_{n=0}^\infty\frac{f^{(n)}(0)z^n}{n!}
      \]\[
        g(z)=\sum_{n=0}^\infty\frac{g^{(n)}(0)z^n}{n!}
      \]
      We also know that
      \[f^{(n)}(0)=\frac{n!}{2i\pi}\left|\int_{C_r}\frac{f(z)}{(z-z_0)^{n+1}}dz\right|\]
      if we choose a small enough $r$ that would mean
      \begin{align*}
        \frac{n!}{2i\pi}\left|\int_{C_r}\frac{f(z)}{(z-z_0)^{n+1}}dz\right|=\frac{n!}{2i\pi}\left|\int_{C_r}\frac{g(z)}{(z-z_0)^{n+1}}dz\right|
      \end{align*}
      Thus:
      \begin{align*}
        f^{(n)}(0)&=g^{(n)}(0)\\
        \implies f(z)&=\sum_{n=0}^\infty\frac{f^{(n)}(0)z^n}{n!}\\
        &=\sum_{n=0}^\infty\frac{g^{(n)}(0)z^n}{n!}\\
        &=g(z)
      \end{align*}
    \item[15]

  \end{enumerate}
\end{document}
