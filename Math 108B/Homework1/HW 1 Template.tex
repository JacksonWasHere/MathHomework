\documentclass{article}
\usepackage[utf8]{inputenc}
\usepackage{amsmath,amsthm,amsfonts}
\usepackage{enumitem} %packages for easier commands
\newcommand{\F}{\mathbb F}
\newcommand{\LL}{\mathcal L}
\newcommand{\M}{\mathcal M}
\newcommand{\N}{\mathbb N}
\renewcommand{\P}{\mathcal P}
\newcommand{\Q}{\mathbb Q}
\newcommand{\R}{\mathbb R}
\newcommand{\Z}{\mathbb Z} %some commands for easier writing

\DeclareMathOperator{\range}{range}
\DeclareMathOperator{\nul}{null}

\newenvironment{problem}[2][Problem]{\begin{trivlist}
\item[\hskip \labelsep {\bfseries #1}\hskip \labelsep {\bfseries #2.}]}{\end{trivlist}}

\newenvironment{solution}
  {\renewcommand{\qedsymbol}{}\begin{proof}[Solution]}
  {\end{proof}}

\title{108B HW 1}
\author{Jackson Weidmann}
\date{\today}


\begin{document}
\maketitle
\section*{3C}
\begin{problem}{2}
Suppose $D\in\LL(\P_3(\R),\P_2(\R))$ is the differentiation map, defined by $Dp=p'$. Find a basis of $\P_3(\R)$ and a basis of $\P_2(\R)$ such that the matrix of $D$ with respect to these bases is
\[\begin{bmatrix} 1 & 0 & 0 & 0\\ 0 & 1 & 0 & 0\\0 & 0 & 1 & 0\end{bmatrix}.\]
\end{problem}
\begin{solution}
  We will give $\P_3(\R)$ the basis \ensuremath{\{x^3,x^2,x,1\}} and $\P_2(\R)$ the basis \ensuremath{\{3x^2,2x,1\}}. If we have some polynomial $p(x)=ax^3+bx^2+cx+d$ we know that $p'(x)=3ax^2+2bx+c$. Rewriting these as matrices we get
  \begin{align*}
    p(x)&=\begin{bmatrix} a\\b\\c\\d \end{bmatrix}\\
    p'(x)&=\begin{bmatrix} a\\b\\c \end{bmatrix}\\
    \implies D\begin{bmatrix} a\\b\\c\\d \end{bmatrix}&=\begin{bmatrix} a\\b\\c \end{bmatrix}
  \end{align*}
\end{solution}
\newpage
\begin{problem}{3}
Suppose $V$ and $W$ are finite-dimensional and $T\in\LL(V,W)$. Prove that there exist a basis of $V$ and a basis of $W$ such that with respect to these bases, all entries of $\M(T)$ are $0$ except that the entries in row $j$, column $j$, equal $1$ for $1\le j\le\dim\range T$.
\end{problem}
\begin{solution}
  First let us suppose that $dim(V)=dim(W)$, and each has basis \ensuremath{\{v_1,v_2,...,v_{n}\}} and \ensuremath{\{w_1,w_2,...,w_{n}\}} respectively. For some $v\in V$
  \begin{align*}
    Iv&=T(v)\\
    \begin{bmatrix}a_1\\a_2\\ \vdots \\a_{n}\end{bmatrix}_V&=\begin{bmatrix}a_1\\a_2\\ \vdots \\a_{n}\end{bmatrix}_W\\
    \implies a_iv_i &= a_iw_i
  \end{align*}
  So they would have the same basis. Assume that $dim(V)>dim(W)$, and each has basis \ensuremath{\{v_1,v_2,...,v_{n}\}} and \ensuremath{\{w_1,w_2,...,w_{m}\}} respectively. For some $v=\begin{bmatrix}a_1\\a_2\\ \vdots \\a_{n}\end{bmatrix}\in V$ we would need that
  \begin{align*}
    M(T)\begin{bmatrix}a_1\\a_2\\ \vdots \\a_{n}\end{bmatrix}_V=\begin{bmatrix}a_1\\a_2\\ \vdots \\a_{m}\end{bmatrix}_W
  \end{align*}
  This would require that the basis of W is the first $m$ vectors (with their first m elements) of the basis for V. If we let $dim(V)<dim(W)$, then we need
  \begin{align*}
    M(T)\begin{bmatrix}a_1\\a_2\\ \vdots \\a_{n}\end{bmatrix}_V=\begin{bmatrix}a_1\\a_2\\ \vdots \\a_{n}\\0\\\vdots\\0\end{bmatrix}_W
  \end{align*}
  As long as the first $n$ vectors of $W$ are the basis vectors of V (with $m-n$ 0 entries at the end) then this will work.
\end{solution}
\newpage
\begin{problem}{4} Suppose $v_1,\ldots,v_m$ is a basis of $V$ and $W$ is finite-dimensional. Suppose $T\in\LL(V,W)$. Prove that there exists a basis $w_1,\ldots,w_n$ of $W$ such that all the entries in the first column of $\M(T)$ (with respect to the bases above) are $0$ except for possibly a $1$ in the first row, first column.
\end{problem}
\begin{solution}
  First let us consider $m\geq n$. Let us consider some \[v=\begin{bmatrix}a_1\\a_2\\ \vdots \\a_{n}\end{bmatrix}=\sum_{i=1}^m\in V\]
  and let's say
  \[T(v)=\begin{bmatrix}b_1\\b_2\\ \vdots \\b_{n}\end{bmatrix}=\sum_{i=1}^nb_iw_i\]
  If we have such an $M(T)$ then
  \begin{align*}
    M(T)\begin{bmatrix}a_1\\a_2\\ \vdots \\a_{n}\end{bmatrix}&=a_1w_1+\sum_{i=2}^nb_iw_i\\
    a_1v_1&=a_1w_1\implies v_1=w_1\\
  \end{align*}
  Thus we need the basis of W to have
\end{solution}
\newpage
\begin{problem}{5}
Suppose $w_1,\ldots,w_n$ is a basis of $W$ and $V$ is finite-dimensional. Suppose $T\in\LL(V,W)$. Prove that there exists a basis $v_1,\ldots,v_m$ of $V$ such that all the entries in the first row of $\M(T)$ are $0$ except for possibly a $1$ in the first row, first column.
\end{problem}
\begin{solution}
\end{solution}
\newpage
\begin{problem}{Additional}
Let $f:\R^2\to \R^2$ be the linear transformation given by
\begin{align*}
f((1,0))&=(-2,1)\\
f((0,1))&=(-1,2).
\end{align*}
Let $T$ be a change of basis for $\R^2$ given by
\begin{align*}
T((1,0))&=(1,1)\\
T((0,1))&=(1,0).
\end{align*}
\begin{enumerate}[label=(\alph*)]
\item Write down the matrix representation $A$ of $f$ under the standard basis $(1,0),(0,1)$.
\item Find the matrix representation $B$ of $f$ under the new basis $(1,1),(1,0)$.
\item Compute the eigenvectors and eigenvalues of $A$ and $B$, respectively.
\item What is the relation between the eigenvalues of $A$ and $B$? Explain.
\item What is the relation between the eigenvectors of $A$ and $B$? Explain.
\end{enumerate}
\end{problem}
\begin{solution}
\end{solution}
\newpage
\section*{5A}
\begin{problem}{7}
  Suppose $T\in\LL(\R^2)$ is defined by $T(x,y)=(-3y,x)$. Find the eigenvalues of $T$.
\end{problem}
\begin{solution}
\end{solution}
\newpage
\begin{problem}{11}
  Define $T:\P(\R)\to\P(\R)$ by $Tp=p'$. Find all eigenvalues and eigenvectors of $T$.
\end{problem}
\begin{solution}
\end{solution}
\newpage
\begin{problem}{21}
  Suppose $T\in\LL(V)$ is invertible.
  \begin{enumerate}[label=(\alph*)]
    \item Suppose $\lambda\in\F$ with $\lambda\neq 0$. Prove that $\lambda$ is an eigenvalue of $T$ if and only if $\frac{1}{\lambda}$ is an eigenvalue of $T^{-1}$.
    \item Prove that $T$ and $T^{-1}$ have the same eigenvectors.
  \end{enumerate}
\end{problem}
\begin{solution}
\end{solution}
\newpage
\begin{problem}{24}
  Suppose $A$ is an $n$-by-$n$ matrix with entries in $\F$. Define $T\in\LL(\F^n)$ by $Tx=Ax$, where elements of $\F^n$ are thought of as $n$-by-$1$ column vectors.
  \begin{enumerate}[label=(\alph*)]
    \item Suppose the sum of the entries of each row of $A$ equals $1$. Prove that $1$ is an eigenvalue of $T$.
    \item Suppose the sum of the entries of each column of $A$ equals $1$. Prove that $1$ is an eigenvalue of $T$.
  \end{enumerate}
\end{problem}
\begin{solution}
\end{solution}
\newpage
\section*{5C}
\begin{problem}{1}
  Suppose $T\in\LL(V)$ is diagonalizable. Prove that $V=\range T\oplus\nul T$.
\end{problem}
\begin{solution}
\end{solution}
\newpage
\begin{problem}{7}
  Suppose $T\in\LL(V)$ has a diagonal matrix $A$ with respect to some basis of $V$ and that $\lambda\in\F$. Prove that $\lambda$ appears in the diagonal of $A$ precisely $\dim E(\lambda,T)$ times.
\end{problem}
\begin{solution}

\end{solution}
\end{document}
